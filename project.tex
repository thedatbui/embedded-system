\documentclass[12pt]{article}
\usepackage[utf8]{inputenc}
\usepackage{hyperref}
\usepackage{csquotes}
\usepackage[backend=biber,style=apa]{biblatex}

\addbibresource{references.bib}

\title{Threat Modeling of HealthKit Applications}
\author{Fernandez Ojeda Franklin {000541971} \\ Bui The Dat {000546997}}
\date{\today}

\begin{document}

\maketitle

\section{Introduction}

Over the past decade, the number of mobile health applications has grown enormously, fueled by rapid technological advancements and the widespread adoption of smartphones. This expansion has transformed the digital health landscape, making mHealth apps central to personal health monitoring and management. At the same time, the growth of these applications has made them increasingly attractive targets for malicious actors.

According to the IQVIA Institute, the global mHealth ecosystem has expanded to more than 350{,}000 digital health applications available on major app stores \parencite{iqvia2021,mobihealth2021}. A similar estimate is reported by the National Institutes of Health (NIH), noting that over 350{,}000 mHealth apps are currently accessible to the public \parencite{ncbi2023}.

Yet this rapid expansion has also significantly broadened the attack surface. Security analyses show systemic weaknesses across the sector: one study revealed that 30 widely used mobile health applications were vulnerable to API-based attacks, exposing full medical records including protected health information (PHI) and personally identifiable information (PII) \parencite{healthleaders2021,securityweek2019}. A more comprehensive industry assessment found that 71\% of mHealth applications contain at least one high-severity vulnerability, while 91\% fail cryptographic security tests \parencite{zimperium2024}.

These findings highlight a critical paradox: although mobile health applications provide substantial value to users and healthcare providers, they also constitute attractive targets for attackers due to the sensitivity of the data they process. Within this ecosystem, Apple’s HealthKit framework stands out as a widely deployed solution that aggregates personal health information, enables fine-grained data sharing, and interacts with third-party apps and cloud services.

This report focuses on performing a comprehensive threat-modeling analysis of applications leveraging HealthKit. By examining the risk landscape and identifying concrete threats, we aim to answer the following central question: Does Apple HealthKit stand out from other mobile health applications in terms of security and what specific threats does it face?

\section{Environment and Trust Zones}

In order to effectively perform threat modeling on HealthKit-based applications, we must first define the environment in which HealthKit operates. This includes identifying actors, system components, data flows, trust boundaries, and the security guarantees associated with each part of the architecture.

\subsection{Actors and Components}

The HealthKit environment consists of several interacting components, each playing a distinct role in the collection, storage, synchronization, and exchange of health-related information. The following subsections describe these actors in detail, with a particular focus on their security properties and their role within the overall trust model.

\subsubsection{User Device (iPhone / Apple Watch}
The primary hardware platform on which HealthKit operates. 
The device stores the entire HealthKit database locally and enforces strict security controls through iOS Data Protection mechanisms \parencite{appleSecurity2023}. 
All health data at rest is cryptographically protected and tied to the user’s device passcode. 
The level of protection depends on whether the device is locked or unlocked, as access to sensitive data is transparently restricted when the device transitions to a locked state. 
From a threat-modeling perspective, the user device represents the core trusted environment.

\subsubsection{HealthKit Store (local)}

The central on-device repository that stores all health and fitness metrics \parencite{appleSecurity2023}. 
This includes step counts, heart-rate samples, workout summaries, sleep analysis, menstrual-cycle data, and more. 
The HealthKit Store is implemented as a secure, encrypted database whose encryption keys are derived from the user’s passcode and device hardware. 
Even Apple cannot decrypt this data. 
The store also manages metadata such as timestamps, data provenance (which app generated each record), and optional cryptographic signatures that ensure data integrity.

\subsubsection{Third-party Health Apps}
External applications that interact with HealthKit via the official APIs \parencite{appleSecurity2023}. 
These apps may request read and/or write access to specific categories of health data, but only after receiving explicit user authorization. 
Access control is enforced through HealthKit entitlements, meaning apps cannot interact with HealthKit unless they are granted permission at both the system and user levels. 
Third-party apps form an important part of the threat landscape since they represent a less-trusted zone with varying degrees of security maturity.

A report in 2021 has shown that third-party health apps using the FHIR standard were vulnerable to hacking due to weak authentication and insecure data transmission. Attackers could exploit these flaws to access millions of patient and clinician records. This highlights that once data leaves a HIPAA-protected environment, it is no longer governed by its stringent security standards.

\subsubsection{External Health Devices}
Devices such as heart-rate monitors, glucose meters, blood-pressure cuffs, activity sensors, and other IoT health peripherals communicating via Bluetooth Low Energy (BLE) \parencite{appleSecurity2021}. 
These devices do not write data directly into HealthKit; instead, they communicate with the paired iPhone or Apple Watch, which then relays the collected measurements through an app. 
Because BLE communication is wireless and device security varies, this constitutes an external data-source zone with its own risks, such as spoofed or tampered sensor data.

\subsubsection{iCloud / Cloud Infrastructure}
An optional synchronization layer enabling users to keep their health data consistent across multiple devices \parencite{applePrivacy2023}. 
If enabled, iCloud can provide end-to-end encryption (E2EE) for health data, meaning only the user’s devices hold decryption keys, and even Apple cannot access the information. 
When E2EE cannot be applied—e.g., if the user has not enabled two-factor authentication—health data remains encrypted in transit and at rest but is not protected end to end. 
iCloud introduces a separate trust boundary since it extends data storage outside the physical device.

\subsubsection{Healthcare Institutions}
Clinical providers, such as hospitals and laboratories, that supply medical records through the Health app’s “Health Records” integration \parencite{appleSecurity2023}. 
The data is obtained using OAuth 2.0 for authentication and transported using TLS 1.3, ensuring strong protection during transmission. 
Once imported, clinical data is treated identically to other HealthKit data and stored in the same secure local database. 
This component represents a semi-trusted external entity with formal, regulated security guarantees.

\subsubsection{Medical ID}
A special subset of emergency information intended for use by first responders \parencite{appleSecurity2023}. 
Unlike standard health data, Medical ID information is accessible directly from the lock screen. 
To enable this, it is stored in a lower data-protection class (“No Protection”), which allows availability even when the device is locked or unpowered. 
From a security-architecture perspective, Medical ID represents a deliberate relaxation of confidentiality constraints to ensure life-critical accessibility.

\subsubsection{Temporary Journals}
A transient storage mechanism used when the device is locked and HealthKit cannot write directly to the encrypted database \parencite{appleSecurity2023}. 
During activities such as workouts, newly generated samples (e.g., heart-rate data) are stored in temporary journal files that remain accessible under the device’s locked state. 
Once the device is unlocked, these journal files are securely merged into the primary HealthKit Store and deleted. 
Temporary Journals constitute a distinct component from a security standpoint because they operate under a weaker protection class and represent a transitional data state.

\subsubsection{Health Sharing (User to user)}
An optional feature introduced in iOS 15 that allows users to share selected health metrics with family members or caregivers \parencite{applePrivacy2023}. 
Health Sharing relies on iCloud’s end-to-end encrypted communication channels, meaning that only the sender and recipient devices hold the encryption keys. 
Even Apple cannot decrypt the shared information. 
This component represents an extension of the trusted environment beyond a single user device, forming a user-to-user trust boundary mediated by secure cloud infrastructure.


\subsection{Data Flows and Trust Boundaries}

Understanding the movement of data within the HealthKit ecosystem is essential for identifying where trust boundaries are established and where potential threats may arise. The following paragraphs describe each data flow in detail, including its security guarantees and associated risks.

\subsubsection{User Device $\longleftrightarrow$ HealthKit Store (Local)}
Communication between the device and the local HealthKit Store represents the core trusted data flow.  
All health data at rest is protected using iOS Data Protection, specifically the ``Protected Unless Open'' class \parencite{appleSecurity2023}.  
When the device locks, the underlying file system enforces cryptographic access restrictions, making sensitive records inaccessible after roughly ten minutes \parencite{appleSecurity2023}.  
This boundary relies entirely on the user’s passcode and hardware-based key derivation, forming the most secure segment of the HealthKit environment.

\subsubsection{HealthKit Store $\longleftrightarrow$ Temporary Journals}
When the device is locked, HealthKit cannot write directly into the encrypted database.  
Instead, new samples—such as heart-rate measurements generated during workouts—are written to temporary journal files \parencite{appleSecurity2023}.  
These files operate under a weaker protection class because they must remain writable while the device is locked.  
Once the device is unlocked, HealthKit automatically merges the journal contents into the main encrypted store and securely deletes the temporary files.  
This trust boundary exposes a transitional state where data confidentiality is still enforced but less strongly than within the fully encrypted database.

\subsubsection{HealthKit Store $\longleftrightarrow$ iCloud}
Health data can be synchronized across devices using iCloud if the user enables this option \parencite{applePrivacy2023}.  
When two-factor authentication and a sufficiently recent OS version (iOS 12+) are in place, the synchronization channel is protected by end-to-end encryption (E2EE), meaning Apple does not hold the decryption keys \parencite{applePrivacy2023}.  
If E2EE cannot be applied, iCloud still encrypts data at rest and in transit, but Apple remains capable of decrypting it under certain conditions \parencite{appleSecurity2023}.  
This introduces a significant trust boundary, as data leaves the physical device and becomes dependent on cloud-level security properties.

\subsubsection{HealthKit $\longleftrightarrow$ Healthcare Institutions}
Clinical records are fetched from hospitals and laboratories using standardized protocols.  
Authentication relies on OAuth 2.0, while data transmission uses TLS 1.3 to ensure confidentiality and integrity \parencite{appleSecurity2021}.  
Once imported, clinical documents are written into the same encrypted HealthKit database as other local health data \parencite{appleSecurity2023}.  
This flow introduces a semi-trusted boundary because it depends on the security posture of external medical providers.
    
\subsubsection{HealthKit $\longleftrightarrow$ Third-Party Apps}
Third-party applications may request read or write access to specific HealthKit data types.  
Such access is tightly controlled through system-level entitlements and user-granted permissions, creating a granular access-control boundary \parencite{appleSecurity2023}.  
Importantly, apps cannot discover whether data exists if they lack permission: HealthKit simply returns an empty result set, preventing metadata-leakage attacks \parencite{applePrivacy2023}.  
Because apps vary widely in security maturity, this boundary constitutes one of the most exposed areas in the HealthKit ecosystem.

\subsubsection{HealthKit $\longleftrightarrow$ Medical ID}
Medical ID information must remain accessible on the lock screen for emergency responders.  
To achieve this, it is stored using the ``No Protection'' class, meaning it is not tied to the device passcode and remains readable even when the phone is locked or powered off \parencite{appleSecurity2023}.  
This is a deliberate weakening of confidentiality guarantees in favor of life-critical availability and introduces a specific, well-understood trust boundary.

\subsubsection{Health Sharing (User to User)}
Starting with iOS 15, users may share selected health metrics with other individuals—such as family members or clinicians—through Health Sharing \parencite{applePrivacy2023}.  
This sharing channel is always protected with iCloud end-to-end encryption, ensuring that only the sender and recipient devices hold the keys.  
Apple explicitly states that it cannot decrypt or access shared data \parencite{applePrivacy2023}.  
This creates an inter-user trust boundary mediated by secure cloud-based E2EE mechanisms, effectively extending HealthKit’s protection beyond the single-device environment.



\subsection{Security Properties and Guarantees}

This subsection details the main security guarantees provided by Apple HealthKit, focusing on data protection, access control, and trust mechanisms.

\subsubsection{Data Protection Classes}
The main HealthKit database uses the ``Protected Unless Open'' class, while management data (e.g., app permissions, connected device names) uses ``Protected Until First User Authentication'' \parencite{appleSecurity2023}.  
These classes define when data is accessible relative to device lock status, ensuring sensitive information is protected when the device is locked.

\subsubsection{Temporal Journals}
Temporary journals store new health entries while the device is locked (e.g., during workouts) and are protected with the same data protection class as the main HealthKit store \parencite{appleSecurity2023}.  
Once the device is unlocked, these journals are securely merged into the main encrypted database, ensuring data continuity without compromising security.

\subsubsection{iCloud Encryption}
Health data synced to iCloud benefits from encryption in transit and at rest, with optional end-to-end encryption (E2EE) if the user has iOS 12+ and two-factor authentication enabled \parencite{applePrivacy2023}.  
Without E2EE, Apple can technically access the encrypted data on its servers, but transport and at-rest encryption still provide a baseline protection.

\subsubsection{Provenance / Authenticity}
Each health record includes metadata indicating the source application, timestamps, and optionally cryptographic signatures using CMS (Cryptographic Message Syntax) \parencite{appleSecurity2023}.  
This ensures data integrity and allows detection of tampering or unauthorized modifications.

\subsubsection{App Access Control}
Third-party apps must request HealthKit entitlements and obtain explicit user permission for each category of health data (e.g., step count, heart rate, sleep analysis, calorie intake, menstrual cycle, workout summaries) \parencite{appleSecurity2023}.  
Access is granular: if a user denies permission for a specific data type, the app will receive an empty result rather than discovering that data exists.  
Permissions are revocable at any time, and apps cannot infer other apps’ granted permissions.  
This mechanism enforces a strict trust boundary between HealthKit and potentially less-trusted third-party applications.

\subsubsection{Medical ID Accessivility}
Emergency medical information is stored with a lower protection class (“No Protection”) to ensure accessibility from the lock screen even when the device is locked \parencite{appleSecurity2023}.  
This is a deliberate trade-off prioritizing life-critical availability over confidentiality.

\subsubsection{User to user Sharing}
HealthKit allows users to share selected health data with other users via iCloud, protected by end-to-end encryption (iOS 15+, 2FA) \parencite{applePrivacy2023}.  
Only the sender and recipient possess the decryption keys, ensuring privacy even from Apple servers.

\section{STRIDE Threat Analysis for Apple HealthKit}

This section applies the STRIDE threat modeling methodology to the Apple HealthKit ecosystem.  
Each threat category is analysed in relation to concrete HealthKit components and data flows, with references to Apple’s official security documentation \parencite{appleSecurity2023,applePrivacy2023,appleSecurity2021} and recent studies on digital health systems \parencite{iqvia2021,mobihealth2021,ncbi2023}.  


Here is the revised text in English, incorporating your requests for a concise overview, corrections, plausible scenarios, and mitigations for each problem.

\subsection{Overview}

HealthKit handles one of the most sensitive categories of personal information. It integrates multiple actors (devices, sensors, apps, clinical institutions) and crosses several trust boundaries (local, cloud, BLE), making STRIDE a relevant framework for identifying security risks.

The STRIDE model categorizes threats as follows: 
\begin{itemize} 
    \item \textbf{Spoofing:} Impersonating a valid user, device, or component. 
    \item \textbf{Tampering:} Maliciously modifying data in transit or at rest. 
    \item \textbf{Repudiation:} Denying that an action was performed. 
    \item \textbf{Information Disclosure:} Exposing data to unauthorized parties. 
    \item \textbf{Denial of Service (DoS):} Preventing legitimate access to the system or data. 
    \item \textbf{Elevation of Privilege:} Gaining capabilities beyond what is authorized. 
\end{itemize}

\newpage \subsection{Detailed STRIDE Analysis} \subsection{Problems related to iCloud} With iOS 12 or later, Apple offers end-to-end encryption (E2EE) as an opt-in feature (Advanced Data Protection), which means Apple cannot access the data. On older versions, or without E2EE enabled, Apple held the encryption keys.

This leads to several risks: 
\begin{itemize} 
    \item A compromise of Apple's servers or a legal request could allow access to these keys and decrypt user health data (\textbf{Information Disclosure}). 
    \item A sophisticated Man-in-the-Middle (MitM) attack targeting Apple's servers was theoretically possible. 
    \item iCloud sync delays can block data consistency, impacting apps or users who need up-to-date information (\textbf{Denial of Service}). 
\end{itemize}

\textbf{Mitigations:} 
\begin{itemize} 
    \item \textbf{User-Side:} Enable Two-Factor Authentication (2FA) and update iOS to enable Advanced Data Protection, which enforces E2EE for most iCloud data, including health backups. 
    \item \textbf{Provider-Side:} Apple's continued migration to E2EE by default. 
\end{itemize}

\subsection{Problems related to Third-party Health Apps} This risk category involves apps abusing the permissions granted by the user, often falling into a regulatory gap (the "HIPAA loophole") where data is no longer protected by health-specific laws once shared by the user.

\begin{itemize} 
    \item An app with write access can inject false data (e.g., a fake heart rate spike) to skew a diagnosis or commit insurance fraud (\textbf{Tampering}). 
    \item A malicious app can impersonate a legitimate fitness or hospital app to trick the user into granting access (\textbf{Spoofing}). 
    \item An app can exfiltrate all health data to a remote server to be sold or used for ad targeting (\textbf{Information Disclosure}). 
    \item An app may deny having written or deleted a health record, creating audit problems (\textbf{Repudiation}). 
    \item Malicious apps could flood HealthKit with junk data, consuming storage and processing power (\textbf{Denial of Service}). 
\end{itemize}

\textbf{Mitigations:} 
\begin{itemize} 
    \item \textbf{Provider-Side:} Apple's strict App Store review, mandatory privacy "nutrition labels," and technical sandboxing of apps. 
    \item \textbf{User-Side:} Scrutinize app permissions before granting, read privacy policies, and only download apps from trusted, well-known developers. 
\end{itemize}

\subsubsection{Sideloading and Malicious Configuration Profiles} "Sideloading" is the process of installing applications from outside the official App Store, typically by tricking a user into installing a malicious configuration profile. 
\begin{itemize} 
    \item \textbf{Method}: An attacker tricks the user into installing an app via an enterprise or developer certificate. 
    \item \textbf{The Risk}: These apps are not verified by Apple. If the user agrees to "Trust" the developer, they are installing malware that can then request and potentially receive HealthKit access. 
\end{itemize}

\textbf{Mitigations:} 
\begin{itemize} 
    \item \textbf{Provider-Side:} OS-level warnings (e.g., "Untrusted Developer"), certificate revocation for known abusers, and (in the EU) notarization requirements for apps outside the App Store. 
    \item \textbf{User-Side:} Never install configuration profiles or "Trust" developers from untrusted sources (e.g., email links, random websites). 
\end{itemize}

\subsection{Problems related to Devices} \subsubsection{External Connected Devices (BLE)} HealthKit relies on Bluetooth Low Energy (BLE) for many sensors. 
\begin{itemize} 
    \item An attacker within physical proximity could intercept or jam BLE communications (\textbf{Denial of Service}). 
    \item A more skilled attacker could impersonate a trusted sensor (e.g., a heart rate monitor) and inject false data (\textbf{Spoofing}, \textbf{Tampering}). 
    \item Exploiting a vulnerability in a poorly designed sensor could potentially allow an attacker to gain unintended permissions (\textbf{Elevation of Privilege}). 
\end{itemize}

\textbf{Mitigations:} 
\begin{itemize} 
    \item \textbf{Provider-Side:} Enforce secure BLE pairing (e.g., authenticated channels) at the OS level. 
    \item \textbf{User-Side:} Use devices from reputable manufacturers that implement secure BLE protocols. 
\end{itemize}

\subsubsection{Device Compromise (Jailbreaking)} A jailbroken device removes all of iOS's fundamental security protections (sandboxing, integrity checks), creating a worst-case scenario.\\

\textbf{Plausible Scenario:} A user takes their device to an unauthorized, "cheap" third-party repair shop. A malicious technician offers to perform a "system optimization" which is actually a jailbreak. The technician then installs a hidden tweak that exfiltrates the HealthKit database (healthdb.sqlite) to a remote server. The user, trusting the "repair," is unaware their entire health history is being stolen.

The direct consequences are: 
\begin{itemize} 
    \item \textbf{Database Theft:} The attacker gains root access and can directly copy the HealthKit database file, bypassing all permissions (\textbf{Elevation of Privilege} \& \textbf{Information Disclosure}). 
    \item \textbf{Data Interception:} Malicious code can be injected directly into the Health app to intercept data in real-time or modify records (\textbf{Tampering}). 
\end{itemize}

\textbf{Mitigations:} 
\begin{itemize} 
    \item \textbf{User-Side:} Do not jailbreak the device; maintain physical control; use only trusted, authorized repair services. 
    \item \textbf{Provider-Side:} Hardware security (Secure Enclave) and software patches to make jailbreaking extremely difficult. 
\end{itemize}

\subsection{Other Problems} \subsubsection{Phishing} An attacker sends an email or SMS impersonating an authority (hospital, insurance, Apple) to trick the user into revealing credentials. \begin{itemize} 
    \item \textbf{Scenario 1 (iCloud Credential Theft)}: The attacker steals the user's Apple ID and password, potentially gaining access to iCloud backups if E2EE is not enabled. 
    \item \textbf{Scenario 2 (Hospital Credential Theft)}: The attacker mimics a hospital portal to steal credentials, then uses them to access the user's official medical records. 
\end{itemize}

\textbf{Mitigations:} 
\begin{itemize} 
    \item \textbf{User-Side:} User education on phishing, enabling 2FA on all accounts (especially Apple ID), and never entering credentials from an email link. 
\end{itemize}

\subsubsection{Physical Access via Medical ID} 
\textbf{Description}: This is an intentional privacy-versus-utility trade-off. The Medical ID is accessible from the lock screen without a passcode so emergency responders can see it. \\
\textbf{The Risk}: Anyone with physical access to a locked phone (a thief, a colleague) can read the sensitive information in the Medical ID (\textbf{Information Disclosure}).\\

\textbf{Mitigations:} 
\begin{itemize} 
    \item \textbf{User-Side:} This is entirely user-controlled. The user can choose what information to include in their Medical ID (e.g., list allergies but not medications) or disable the "Show When Locked" feature entirely. 
\end{itemize}

\subsubsection{Supply Chain Attacks} \textbf{Description}: This is a sophisticated attack that targets the app developer, not the end-user. An attacker compromises a third-party tool, such as an analytics SDK or ad library, that the developer uses. \\
\textbf{The Risk}: The developer unknowingly embeds this malicious code into their own legitimate app. This "Trojan" app then passes App Store review but contains a hidden backdoor to exfiltrate HealthKit data from all of its users (\textbf{Information Disclosure}, \textbf{Tampering}, \textbf{Elevation of Privilege}).

\textbf{Mitigations:} 
\begin{itemize} 
    \item \textbf{Provider-Side (Apple):} Continued enhancement of static and dynamic analysis during App Store review to detect malicious behavior in third-party SDKs. 
    \item \textbf{Provider-Side (Developer):} Vetting all third-party dependencies; using secure coding practices; minimizing the dependencies included in their app. 
\end{itemize}


\section{Conclusion}
To answer the core question if Apple HealthKit stands out from other mobile health applications in terms of security HealthKit provides one of the most robust foundations available. Its security is built on a locally encrypted database (using keys the user controls, which Apple cannot decrypt) and offers opt-in, end-to-end encryption for iCloud backups via Advanced Data Protection, ensuring a high level of privacy.

However, as demonstrated in the STRIDE analysis, this architecture doesn't eliminate all threats. Two main responsibilities must be considered: \begin{itemize} \item \textbf{Apple} is responsible for maintaining the integrity of the platform. This includes patching vulnerabilities (such as historical certificate abuse and app-spoofing), strictly vetting the App Store to enforce privacy rules, and providing users with tools like Advanced Data Protection for end-to-end cloud encryption. \item \textbf{Users} must be responsible. Even if HealthKit is one of the most secure apps, it can still be compromised by user actions, such as granting excessive permissions to untrusted third-party apps, falling victim to sophisticated phishing attacks, or choosing a weak device passcode. \end{itemize}

In conclusion, HealthKit does stand out by providing a high-security baseline that mitigates entire classes of threats by default. However, the practical, day-to-day security of the data rests on a dynamic partnership between Apple's robust engineering and the user's vigilant awareness.
\section{Biblio}
\printbibliography

\end{document}
